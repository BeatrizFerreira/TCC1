\begin{resumo}[Abstract]
 \begin{otherlanguage*}{english}

The development of new technologies has collaborated for innovation on software development. Those software are more complex and robust. In order to fulfill this need, software systems began to be composed by subsystems or modules. These subsystems are characterized by providing their functionality as a service to the larger systems. This can be seen in banking systems, where some modules are legacy systems while others are modern systems. This type of system can be built with usage of architectural approach called SOA, or Service Oriented Architecture. This project aims to build an architecture using this model to integrate applications resulting from TCC guidance and activities under a software development and research laboratory in the University of Brasilia. The compilation of the outcomes of these works leads to a heterogeneous system with web platform characteristics. The construction of this platform based on SOA model uses a service bus, where applications and services are connected. This service bus is responsible for data routing, formatting and processing. Those data are exchanged via messaging between the architecture components.

   \vspace{\onelineskip}
 
   \noindent 
   \textbf{Key-words}: Software Architecture. Service-Oriented Architecture. Integrated Virtual Environment. Heterogeneous Systems. Service Bus.
 \end{otherlanguage*}
\end{resumo}
