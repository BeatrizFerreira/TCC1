\chapter[Introdução]{Introdução}
Arquitetura de software é, segundo Bass, Clements e Kasman \cite{bass_software_archi_practice_2003}, um conjunto de estruturas que representam os componentes de um software e a maneira como tais componentes se relacionam. Todo sistema de software possui uma arquitetura, mesmo que não definida ou documentada \cite{bass_software_archi_practice_2003}.

A maneira como os componentes de um \textit{software} são dispostos e combinados para que a melhor solução seja construída dá origem a estilos arquiteturais \cite{pressman2006engenharia}. Pode-se citar os estilos baseado em camadas, eventos e serviços, sendo este último mais conhecido como SOA \cite{josuttis_soa_2007}. Este estilo baseado em serviços (SOA) é amplamente utilizado quando o sistema de software a ser construído é composto de outros sistemas que oferecem suas funcionalidades como um serviço. O estilo arquitetural SOA facilita a implementação da interoperabilidade em um sistema, permitindo a troca de dados e interação entre aplicações que utilizam tecnologias distintas \cite{oqueesoa_2010}.

Por ser um estilo arquitetural indicado para a construção de sistemas heterogêneos e distribuídos \cite{josuttis_soa_2007}, a implementação de um sistema baseado em serviços deve incorporar diferentes tecnologias, APIs e composições de infraestrutura, resultando sempre em um produto de arquitetura única \cite{erl_orientacaoaservico_2009}.

\section{Motivação}
Alguns trabalhos oriundos de projetos de TCC no curso de Engenharia de \textit{Software} são realizados e têm uma aplicação de \textit{software} como produto final. Geralmente, estes sistemas de \textit{software} são construídos visando solucionar um problema ou uma necessidade que foi identificada. Contudo, estas ideias acabam sendo esquecidas ou desenvolvidas de modo incompleto por diversos motivos, não podendo ser utilizadas.

Baseando-se neste contexto, a principal motivação para a realização deste projeto de TCC foi proporcionar um produto capaz de dar maior visibilidade e utilidade de sistemas de \textit{software} que são desenvolvidos em atividades práticas do curso de Engenharia de \textit{Software}. Desta maneira, os \textit{software} produzidos podem ser utilizados pela comunidade acadêmica. Consequentemente, um número maior de aplicações resultantes de trabalhos de TCC na Engenharia de Software será conhecido, mantido e evoluído. 

\section{Objetivos}

Nesta seção, os objetivos geral e os específicos deste projeto são apresentados, bem como a questão de pesquisa que guia este TCC.

\subsection{Objetivo Geral}
O objetivo geral é especificar uma arquitetura de \textit{software} que permita a integração entre diversas aplicações de \textit{software}, implementando a troca de dados entre estes. Estas aplicações fazem parte dos resultados de trabalhos do grupo de orientandos e dos trabalhos desenvolvidos em laboratórios de pesquisa e desenvolvimento de \textit{software} da Universidade de Brasília.

\subsection{Objetivos específicos}
Os objetivos específicos deste trabalho são:
\begin{itemize}
\item Disponibilizar à comunidade acadêmica um ambiente virtual composto por aplicações desenvolvidas em projetos de TCC.
\item Integrar sistemas heterogêneos em uma plataforma unificada.
\item Propor e desenvolver uma arquitetura baseada no modelo SOA para um ambiente virtual.
\item Definir um protocolo de comunicação para troca de dados entre as aplicações que compõem o ambiente virtual.
\end{itemize}

\subsection{Questão de pesquisa}
A questão de pesquisa que move este trabalho é: "Como é possível implementar um sistema de \textit{software} composto a partir da integração de aplicações individuais, sendo estas aplicações desenvolvidas com base em tecnologias, técnicas e métodos distintos?"

A questão secundária é: "As aplicações que compõem este sistema podem ser executadas de modo \textit{standalone} ou apenas em conjunto ao sistema?"


\section{Metodologia}
A metodologia usada para o desenvolvimento deste trabalho está descrita nesta seção. 

\subsection{Classificação da pesquisa}
De acordo com Gil \cite{gil_como_2008}, as pesquisas podem ser classificadas de acordo com os objetivos e procedimentos técnicos para sua realização. Quanto aos objetivos, podem ser classificadas em: exploratória, descritiva e explicativa. 

O presente projeto de TCC classifica-se como uma pesquisa exploratória. Este tipo de pesquisa visa maior conhecimento e domínio sobre o problema ou a necessidade a ser investigada durante sua execução. Como procedimento técnico, pode adotar métodos que a classificam como uma pesquisa bibliográfica ou um estudo de caso na maioria das vezes \cite{gil_como_2008}.

\subsection{Referencial teórico}
Para a proposição de uma arquitetura necessária para atingir os objetivos deste projeto de TCC, a pesquisa acerca do referencial teórico foi realizada por meio da busca de livros-textos sobre o assunto e artigos publicados sobre implementações já realizadas. Estes livros-textos foram encontrados em meios físico e digital. A compilação desta pesquisa encontra-se no capítulo 2 (Referencial Teórico).

\section{Estrutura da Monografia}
O presente documento está dividido em quatro partes: Referencial Teórico, A Arquitetura Implementada, Desenvolvimento da Arquitetura e Considerações Finais. O capítulo de referencial teórico tem como finalidade expor os estudos realizados e os conceitos relacionados a arquitetura implementada. O terceiro capítulo, A Arquitetura Implementada, contém detalhes sobre a proposta do ambiente integrativo deste projeto de TCC. A metodologia relacionada à execução da proposta, bem como testes e resultados são detalhados no capítulo quatro (Desenvolvimento da Arquitetura). No quinto e último capítulo, nomeado Considerações Finais, são expostas as conclusões e trabalhos futuros.