

\chapter[Desenvolvimento da Proposta]{Desenvolvimento da Proposta}

%\section{Introdução}



\section{Descrição do projeto}


\section{Proposta}

% O sistema desenvolvido para plataforma web, aonde um tutor consiga gerenciar seus exames de questões dissertativas, assim sendo, ele cadastrará um exame, cadastrará perguntas e gabaritos ao exame, disponibilizará para receber respostas, corrigirá exames, disponibilizar notas e corrigirá recursos, a parti do gabarito da resposta, que foi inserido usando a sintaxe de marcação, usará processamento de linguagem natural, algoritmos de similaridade e algoritmo genético, para determinar uma nota para a reposta, com a interação com o tutor, o sistema irá melhora a função da pré-avaliação, com isso se espera demandar menos tempo de correção pelo tutor.


\section{O sistema}

\section{Definição das tecnologias}

\section{Cronograma de desenvolvimento}

Para o desenvolvimento da proposta, foi criado o cronograma \ref{cronograma_tcc1}, onde estão descritas as atividades realizadas, bem como os prazos relacionados a cada atividade.

\begin{table}[!h]
\centering
\caption{Cronograma de atividades relacionadas ao TCC 1}
\label{cronograma_tcc1}
\begin{tabular}{|l|c|c|c|c|c|c|}
\hline
Atividade                                                   & \multicolumn{1}{l|}{Jan} & \multicolumn{1}{l|}{Fev} & \multicolumn{1}{l|}{Mar} & \multicolumn{1}{l|}{Abr} & \multicolumn{1}{l|}{May} & \multicolumn{1}{l|}{Jun} \\ \hline
Identificação de um problema a ser solucionado              & X                           &                             &                              &                            &                             &                              \\ \hline
Leituras sobre padrões arquiteturais                        & X                           &                             &                              &                            &                             &                              \\ \hline
Definição da proposta de arquitetura                        & X                           &                             &                              &                            &                             &                              \\ \hline
Pesquisa sobre trabalhos relacionados                       & X                           & X                        &                              &                            &                             &                              \\ \hline
Análise e escolha de ferramentas                            &                             & X                        & X                            &                            &                             &                              \\ \hline
Organização das referências                                 &                             &                             & X                            &                              \\ \hline
Escrita do TCC 1 Capítulo 3-Proposta                        &                             &                             &                              & X                            \\ \hline
Escrita do TCC 1 Capítulo 2-Referencial teórico             &                             &                             &                              & X                            \\ \hline
Escrita do TCC 1 Capítulo 4-Engenharia de Software          &                             &                             &                              & X                            \\ \hline
Escrita do TCC 1 Capítulo 1-Introdução                      &                             &                             &                              & X                            \\ \hline
\end{tabular}
\end{table}

