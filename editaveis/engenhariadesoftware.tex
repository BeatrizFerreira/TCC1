\chapter[Desenvolvimento da Proposta]{Desenvolvimento da Proposta}

Este capítulo apresenta o uso de metodologias e conceitos relacionados à Engenharia de Software que serão aplicados no desenvolvimento da proposta descrita no capítulo anterior.

\section{Introdução}

Um projeto de Engenharia de Software deve ser realizado utilizando-se de metodologias e técnicas relacionadas à esta área de conhecimento, sendo sempre adaptadas de acordo com o projeto a ser desenvolvido e podendo ser modificado quando necessário, visando o sucesso na conclusão do projeto.

O projeto de Engenharia de Software a ser desenvolvido como parte do trabalho de conclusão de curso consiste em uma arquitetura para uma plataforma virtual onde as funcionalidades serão tratadas como serviços no contexto arquitetural. Os serviços ou funcionalidades desta plataforma virtual consiste em trabalhos já realizados, em processo de desenvolvimento e também de trabalhos futuros que, quando incorporados a esta arquitetura, poderão ser disponibilizados para uso pela comunidade tanto acadêmica quando externa, deixando que ser "projetos de gaveta".

A proposta feita, que foi descrita no capítulo anterior, será desenvolvida utilizando-se de alguns conceitos de metodologias de desenvolvimento e de gerenciamento de projetos de software, adaptadas conforme as necessidades deste projeto.

%COLOCAR AQUI A METODOLOGIA A SER UTILIZADA PARA A IMPLEMENTAÇÃO DA ARQUITETURA. VERIFICAR QUAL A METODOLOGIA MAIS ADEQUADA PARA EXECUÇÃO E GERENCIAMENTO. 
%Sugerido: Scrum com RUP incremental (iniciação - ja feita, elaboração, execução e implantação).
%RUP incremental pq podemos fazer parte da adaptação de um serviço, e colocar na plataforma; outra parte, e acoplar à plataforma e assim por diante. Pensar também no layout e tentar implementar o login (login pode ser implementado, mas não afirmando que isso vai ser feito, deixar em off o login).

\section{Metodologia de desenvolvimento}

Dentre as diversas metodologias de desenvolvimento conhecidas na Engenharia de Software, foi decidido pela elaboração de uma metodologia que utiliza as fases do RUP (iniciação, elaboração, construção e transição) e algumas das atividades também deste método de desenvolvimento de software. Além disso, o método adotado terá um ciclo de vida incremental, onde partes da arquitetura serão desenvolvidas e incrementadas a cada ciclo.

Por se tratar de um projeto mais técnico do que funcional, a adoção de uma outra metodologia de desenvolvimento, como a metodologia ágil, tornou-se complexa para a identificação de unidades funcionais unitárias, como histórias de usuário (um conceito da metodologia ágil utilizado para descrever as funcionalidades de um sistema de software), o estabelecimento de ciclos bem definidos e a distribuição das unidades funcionais neste ciclo. Isto se deve ao fato de que o estabelecimento da arquitetura deve ser realizado em conjunto com o uso do protocolo de comunicação de acordo com o padrão estabelecido e isto deve ser realizado de maneira incremental, pois estas partes são dependentes uma da outra.

\subsection{Fases de desenvolvimento}

Adotando o RUP como metodologia de desenvolvimento, as fases a serem executadas são:

\begin{itemize}
\item \textit{Iniciação}: fase de concepção do projeto, onde as necessidades são identificadas e a solução é proposta.
\item \textit{Elaboração}: fase em que é realizado o planejamento da solução proposta. Também são identificados os recursos necessários, tanto de pessoal quanto de infraestrutura física.
\item \textit{Construção}: fase em que os requisitos identificados são de fato desenvolvidos/construídos. Também nesta fase são realizados testes a fim de garantir a menor taxa de erros no sistema.
\item \textit{Transição}: fase em que o software é homologado e implantado no ambiente do cliente, dono do software.
\end{itemize}

Os ciclos de desenvolvimento adotados serão de forma incremental de maneira a adaptar parte de um sofwtare desenvolvido em trabalhos anteriores para que este seja incorporado a plataforma virtual como um serviço e a comunicação entre a plataforma virtual e este serviço esteja de acordo com o padrão de comunicação adotado. O objetivo principal desta iteração é garantir que exista um modelo prático a ser seguido quando novas aplicações forem desenvolvidas e incorporadas à plataforma virtual.

---DEFINIR QUANTIDADE DE CICLOS A SEREM RODADOS---
---COLOCAR UMA FIGURA EXPLICANDO O CICLO DE DESENVOLVIMENTO---
---O CICLO ADOTADO DEVERÁ CONTER UMA GRANDE FASE DE INICIAÇÃO E ELABORAÇÃO NO PRIMEIRO CICLO. O SEGUNDO, TERCEIRO E QUARTO CICLOS DEVERÃO SER DE 7 A 10 DIAS. NESTES CICLOS, A FASE DE INICIAÇÃO DEVE SER BEM PEQUENA, SÓ INDICANDO DE NOVAS NECESSIDADES PODEM SER IDENTIFICADAS; A FASE DE ELABORAÇÃO DEVE SER UM POUCO MAIOR QUE A INICIAÇÃO, INDICANDO O PLANEJAMENTO DO NOVO CICLO E A IDENTIFICAÇÃO DE NECESSIDADE DE NOVOS RECURSOS DE HARDWARE; A FASE DE CONSTRUÇÃO DEVE INDICAR IMPLEMENTAÇÃO DA PLATAFORMA, ADAPTAÇÃO DO SOFTWARE E TESTES. A IMPLANTAÇÃO DEVERÁ SER FEITA AO FINAL  DOS CICLOS, E TAMBÉM DEVE INDICAR A EXISTÊNCIA DE TESTES---
------


A imagem acima mostra o ciclo de desenvolvimento a ser adotado com base nas fases do modelo RUP. A imagem mostra as fases de iniciação, elaboração, construção e implantação a serem executadas durante o trabalho de conclusão de curso. A parte incremental é realizada durante a construção da arquitetura, onde pequenos ciclos são executados e o foco maior está na fase de construção.

A fase de construção consiste em ciclos de desenvolvimento onde os requisitos e necessidades serão revistos e implementados usando-se de tecnologias mais atuais e adequadas ao contexto. Este ciclo é exibido desta maneira na figura a fim de demonstrar que, embora as fases de iniciaão, elaboração e implantação tenham seus períodos definidos e maior parte dos trabalhos de identificação das necessidades, planejamento e homologação da arquitetura proposta sejam realizados em suas respectivas fases, estas tarefas também devem ser realizadas durante a construção, principalmente se o serviço que for incorporado à arquitetura tiver que ser adaptado para tal finalidade. Para que as aplicações sejam adaptadas, as fases indicadas devem ser executadas de maneira cíclica e incremental, de modo que as necessidades de adaptação sejam identificadas e depois implementadas. A implantação ao final de cada ciclo será um teste para verificar e validar a adequação realizada.

A execução destes ciclos menores se faz necessário visto que, como existem aplicações já desenvolvidas, um dos objetivos a serem alcançados ao final do TCC 2 é a adaptação e implantação de uma destas aplicações e torná-la um serviço dentro da plataforma virtual.

\section{Cronograma de execução}
Para o desenvolvimento da proposta e completude da mesma com sucesso foi criado um cronograma para a fase de prova de conceito e implementação da proposta aqui feita. O cronograma contém a descrição das atividades a serem realizadas, bem como os prazos relacionados a cada uma das atividades e pode ser visualizado na tabela abaixo.

\begin{table}[!h]
\centering
\caption{Cronograma de atividades relacionadas ao TCC 2}
\label{cronograma_tcc2}
\begin{tabular}{|p{9cm}|c|c|c|c|c|c|}
\hline
Atividade                                                   & \multicolumn{1}{l|}{Jul} & \multicolumn{1}{l|}{Ago} & \multicolumn{1}{l|}{Set} & \multicolumn{1}{l|}{Out} & \multicolumn{1}{l|}{Nov} & \multicolumn{1}{l|}{Dez} \\ \hline
Análise de ferramentas a serem utilizadas                   & X                           &                             &                              &                            &                             &                              \\ \hline
Implantação da ferramenta escolhida                         &                             & X                            &                              &                            &                             &                              \\ \hline
Adaptação de uma aplicação já desenvolvida                  &                             &                               & X                            & X                          &                             &                              \\ \hline
Implementação do uso da ferramenta com a aplicação adaptada &                             &                         & X                            & X                          &                             &                              \\ \hline
Escrita do TCC 2                                            &                             &                             &                              &                            & X                           & X          \\ \hline
\end{tabular}
\end{table}



--- TEM QUE INDICAR NO CRONOGRAMA OU EXPLICAR EM FORMA DE TEXTO QUE OS CICLOS VÃO SER EXECUTADOS  FURANTE AGOSTO E SETEMBRO (OU É SETEMBRO E OUTUBRO) E QUE A CARACTERIZAÇÃO DAS FERRAMENTAS esb FAZEM PARTE DA ELABORAÇÃO/INICIAÇÃO (VERIFICAR ISSO, CHECAR AS FASES DO RUP)---


As atividades realizadas durante o desenvolvimento do TCC 1 foram registradas e podem ser visualizadas na tabele seguinte.


\begin{table}[!h]
\centering
\caption{Cronograma de atividades relacionadas ao TCC 1}
\label{cronograma_tcc1}
\begin{tabular}{|p{9cm}|c|c|c|c|c|c|}
\hline
Atividade                                                   & \multicolumn{1}{l|}{Jan} & \multicolumn{1}{l|}{Fev} & \multicolumn{1}{l|}{Mar} & \multicolumn{1}{l|}{Abr} & \multicolumn{1}{l|}{Mai} & \multicolumn{1}{l|}{Jun} \\ \hline
Identificação da necessidade                                & X                           &                             &                              &                            &                             &                              \\ \hline
Leituras sobre modelos arquiteturais                        & X                           &                             &                              &                            &                             &                              \\ \hline
Pesquisa sobre o modelo arquitetural mais adequado          & X                           & X                             &                              &                            &                             &                              \\ \hline
Pesquisa sobre trabalhos relacionados                       & X                           & X                        &                              &                            &                             &                              \\ \hline
Levantamento superficial de ferramentas ESB                 &                             & X                        & X                            &                            &                             &                              \\ \hline
Organização das referências                                 &                             &                             & X                            &                            &                             &           \\ \hline
Escrita formal da proposta                                  &                             &                             &                              & X                          &                             &           \\ \hline
Estabelecimento de metodologia                              &                             &                             &                              & X                          &                             &             \\ \hline
Escrita do TCC 1                                            &                             &                             &                              &                            & X                           & X          \\ \hline
\end{tabular}
\end{table}
