

\chapter[Desenvolvimento da Proposta]{Desenvolvimento da Proposta}

\section{Introdução}

COLOCAR AQUI A METODOLOGIA A SER UTILIZADA PARA A IMPLEMENTAÇÃO DA ARQUITETURA. VERIFICAR QUAL A METODOLOGIA MAIS ADEQUADA PARA EXECUÇÃO E GERENCIAMENTO. 
Sugerido: Scrum com RUP incremental (iniciação - ja feita, elaboração, execução e implantação).
RUP incremental pq podemos fazer parte da adaptação de um serviço, e colocar na plataforma; outra parte, e acoplar à plataforma e assim por diante. Pensar também no layout e tentar implementar o login (login pode ser implementado, mas não afirmando que isso vai ser feito, deixar em off o login).

\section{Proposta}

% O sistema desenvolvido para plataforma web, aonde um tutor consiga gerenciar seus exames de questões dissertativas, assim sendo, ele cadastrará um exame, cadastrará perguntas e gabaritos ao exame, disponibilizará para receber respostas, corrigirá exames, disponibilizar notas e corrigirá recursos, a parti do gabarito da resposta, que foi inserido usando a sintaxe de marcação, usará processamento de linguagem natural, algoritmos de similaridade e algoritmo genético, para determinar uma nota para a reposta, com a interação com o tutor, o sistema irá melhora a função da pré-avaliação, com isso se espera demandar menos tempo de correção pelo tutor.


\section{Metodologia de desenvolvimento}

\section{Cronograma de execução}
Para o desenvolvimento da proposta e completude da mesma com sucesso foi criado um cronograma para a fase de prova de conceito e implementação da proposta aqui feita. O cronograma contém a descrição das atividades a serem realizadas, bem como os prazos relacionados a cada uma das atividades e pode ser visualizado na tabela abaixo.

\begin{table}[!h]
\centering
\caption{Cronograma de atividades relacionadas ao TCC 2}
\label{cronograma_tcc2}
\begin{tabular}{|p{9cm}|c|c|c|c|c|c|}
\hline
Atividade                                                   & \multicolumn{1}{l|}{Jul} & \multicolumn{1}{l|}{Ago} & \multicolumn{1}{l|}{Set} & \multicolumn{1}{l|}{Out} & \multicolumn{1}{l|}{Nov} & \multicolumn{1}{l|}{Dez} \\ \hline
Análise de ferramentas a serem utilizadas                   & X                           &                             &                              &                            &                             &                              \\ \hline
Implantação da ferramenta escolhida                         &                             & X                            &                              &                            &                             &                              \\ \hline
Adaptação de uma aplicação já desenvolvida                  &                             &                               & X                            &                            &                             &                              \\ \hline
Implementação do uso da ferramenta com a aplicação adaptada &                             &                         &                              & X                          &                             &                              \\ \hline
Escrita do TCC 2                                            &                             &                             &                              &                            & X                           & X          \\ \hline
\end{tabular}
\end{table}


As atividades realizadas durante o desenvolvimento do TCC 1 foram registradas e podem ser visualizadas na tabele seguinte.


\begin{table}[!h]
\centering
\caption{Cronograma de atividades relacionadas ao TCC 1}
\label{cronograma_tcc1}
\begin{tabular}{|p{9cm}|c|c|c|c|c|c|}
\hline
Atividade                                                   & \multicolumn{1}{l|}{Jan} & \multicolumn{1}{l|}{Fev} & \multicolumn{1}{l|}{Mar} & \multicolumn{1}{l|}{Abr} & \multicolumn{1}{l|}{Mai} & \multicolumn{1}{l|}{Jun} \\ \hline
Identificação da necessidade                                & X                           &                             &                              &                            &                             &                              \\ \hline
Leituras sobre modelos arquiteturais                        & X                           &                             &                              &                            &                             &                              \\ \hline
Pesquisa sobre o modelo arquitetural mais adequado          & X                           & X                             &                              &                            &                             &                              \\ \hline
Pesquisa sobre trabalhos relacionados                       & X                           & X                        &                              &                            &                             &                              \\ \hline
Levantamento superficial de ferramentas ESB                 &                             & X                        & X                            &                            &                             &                              \\ \hline
Organização das referências                                 &                             &                             & X                            &                            &                             &           \\ \hline
Escrita formal da proposta                                  &                             &                             &                              & X                          &                             &           \\ \hline
Estabelecimento de metodologia                              &                             &                             &                              & X                          &                             &             \\ \hline
Escrita do TCC 1                                            &                             &                             &                              &                            & X                           & X          \\ \hline
\end{tabular}
\end{table}
