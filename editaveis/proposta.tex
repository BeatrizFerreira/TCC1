\chapter[A PROPOSTA]{A PROPOSTA}

Este capítulo apresenta a proposta do trabalho de conclusão de curso, com detalhes sobre [...]

\section{Introdução}
Avanços tecnológicos e a criação de linguagens de programação, diferentes técnicas e paradigmas e outros conceitos relacionados ao desenvolvimento de software contribui para que a necessidade de interação entre estes elementos seja emergente, de modo a viabilizar a construção de sistemas cada vez mais robustos e inteligentes. Esta interação entre elementos de software não consistem de aplicações robustas que executam todas as suas atividades de forma independente de outras aplicações: cada vez mais, diversos sistemas de software interagem com outros, podendo ser estes desenvolvidos tomando como base outros paradigmas ou escritos em outras linguagens de programação e utilizando-se diferentes técnicas.

A fim de suprir esta necessidade de interação entre sistemas diversos, foi criado um modelo arquitetural conhecido como Arquitetura Baseada em Serviços (ou \textit{Service-Oriented Architecture} - SOA). Este modelo arquitetural utiliza o conceito de serviço como uma unidade que representa uma funcionalidade do sistema, além de trazer consigo os conceitos de interoperabilidade, flexibilidade e baixo acoplamento entre os diversos sistemas ou serviços.

Para este trabalho de conclusão de curso, a proposta consiste em desenvolver uma arquitetura baseada no modelo SOA para um ambiente virtual, propiciando que diversos trabalhos já desenvolvidos e também aqueles em desenvolvimento não sejam mais "engavetados". Por meio do uso do modelo SOA, será possível integrar tais aplicações, ou serviços nos moldes dos termos de SOA, de modo que estas possam trocar dados e fazer uso do serviço disponibilizado por outras, indepentemente das tecnologias utilizadas para o desenvolvimento das mesmas. Também faz parte da proposta, o estabelecimento de um protocolo de comunicação entre as aplicações, bem como o padrão de comunicação a ser utilizado.

Desta forma, será viável a disponibilização à sociedade de uma plataforma virtual que conterá os trabalhos realizados dentro da universidade, sejam oriundos de projetos de conclusão de curso, disciplinas ou projetos de extensão e pesquisa.

\section{Proposta de arquitetura}

\subsection{Modelo de Domínio}

{REESCREVER O MODELO DE DOMINIO, DE FORMA A SER MAIS AMPLO E NÃO SOMENTE RESTRITO AO CONTEXTO EDUCACIONAL}

\subsection{Requisitos}

{FALAR AQUI SOBRE OS REQUISITOS LEVANTADOS E DESCRITOS ANTERIORMENTE}

\subsection{A arquitetura}

{FALAR AQUI COMO SERÁ A ARQUITETURA, CONTAR DETALHES, MOSTRAR FIGURAS, PROPOR CENÁRIO DE APLICAÇÃO E JUSTIFICAR OS DETALHES QUE FIZERAM A ESCOLHA PELO MODELO SOA E NÃO POR OUTRO QUALQUER}

\subsection{Uso do ESB}

Como já mencionado, o Enterprise Service Bus - ESB - é um conceito que está diretamente conectado a SOA e é um mecanismo que pode ser utilizado para ajudar a prover a interoperabilidade na arquitetura estabelecida, além de colaborar na validação de algumas políticas e critérios definidos relacionados à segurança da aplicação baseada neste modelo arquitetural \cite{oliveira_interoperabilidade}.

Para facilitar o processo de roteamento, bem como de transformação de mensagens trocadas entre provedores e usuários dos serviços, faz-se necessário o uso do ESB. Ao adicionar um novo serviço à arquitetura utilizando o ESB, deverão ser especificados tanto os procedimentos quando uma nova requisição for feita quanto aqueles passos que conduzem ao envio de uma resposta de acesso ao serviço especificado, sendo esta resposta advinda do próprio serviço ou uma resposta padrão, também especificada, em caso de falha ou sucesso ao executar a requisição.

--COLOCAR AQUI UMA FIGURA--

A figura acima elucida o papel do ESB na proposta de arquitetura a ser implementada.

\subsubsection{Ferramentas ESB}
Existem diversas ferramentas deste tipo disponíveis e em uso por grandes organizações, tais como JBoss ESB, Mule ESB, Zato e WSO2 ESB. Algumas destas ferramentas já foram levantadas, e, sendo o ESB um elemento importante para a implementação aqui porposta, uma análise destas ferramentas com junto a uma avaliação baseada em critérios ainda não estebelecidos a fim de eleger uma ferramenta deste tipo para apoiar o desenvolvimento do trabalho.

\subsection{Protocolo de comunicação}

{DESCREVER O PROTOCOLO, JUSTIFICAR O USO E ESTABELECER OS PADRÕES A SEREM UTILIZADOS}

\section{Metodologia}


\section{Cronograma de execução}
