

\chapter[A PROPOSTA]{A PROPOSTA}

"Neste capítulo é descrita a proposta do trabalho".		


\section{Introdução}
%Com a evolução da tecnologia, os computadores se tornaram cada vez mais parte do nosso cotidiano, com as mais diversas utilidades, desde cálculos complexos e previsão do tempo, diversão, comunicação, educação entre outros, com isso o computador se tornou algo fundamental na sociedade.

%Para a educação a utilização cada vez maior de Ambientes virtuais de aprendizado (AVA), que podem ser usados tanto para alunos de cursos presenciais ou de educação a distância (EAD), melhora o acompanhamento pelo professor, que tem a possibilidade de oferecer as matérias de estudo, recolher atividades, aplicar testes entre outras atividades oferecidas.

%Nesse trabalho iremos focar a parte de avaliação, nas AVA's podem-se ter avaliações por meio de perguntas de múltipla escolha, verdadeiro ou falso entre outros, a avaliação de questões discursivas pode ser realizada, mas sua correção tem que ser feita manualmente, assim com processamento de linguagem natural (PLN), uma subárea da inteligência artificial, poderá ser usada para a realização de uma pré-avaliação de questões discursivas de forma a guiar os tutores na correção, que ele poderá aceitar ou dar a nota que ele achar correta.

%A proposta é desenvolver uma forma de analisar a resposta discursiva com um gabarito definido, com algoritmos de similaridade de texto como \textit{latent semantic analysis} (LSA) e STASIS, podemos comprará o quanto que dois textos são similares e através dessa ter uma nota de similaridade, além dessa análise, uma vez que o tutor der uma nota que não seja a proposta pelo sistema, um algoritmo genético irá analisar quais elementos do texto implicou na diferença na nota, com essa diferença armazenada o sistema aprende qual parte do texto deverá ter maior ou menor nota e quais partes do texto não fazem parte do gabarito inicialmente definido que também alteram a nota final, para que na próxima vez que for aplicada essa pergunta já ter novos pesos.

%Para o texto gabarito foi proposto uma linguagem de marcação, que permitisse marcar partes mais importantes do texto gabarito, definir que alguma palavra pode ser substituída por um sinônimo, tesauro ou metonímia e marcar palavras que serão definidas como negativas há nota final. A definição da sintaxe se encontra no apêndice 1, Definição da sintaxe da definição de gabarito.
