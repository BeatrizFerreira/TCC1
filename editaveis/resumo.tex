\begin{resumo}

O desenvolvimento tecnológico tem colaborado e exigido a construção de sistemas de \textit{software} cada vez mais robustos e complexos. A fim de atender à esta demanda, foram criados mecanismos, nos quais os sistemas são compostos por módulos ou subsistemas de \textit{software}. Estes subsistemas caracterizam-se por fornecer suas funcionalidades como serviços ao sistema maior. Isto pode ser visto em sistemas bancários, onde alguns módulos são sistemas legados, enquanto outros são sistemas mais modernos. Este tipo de sistema pode ser construído a partir do uso da abordagem arquitetural denominada SOA, ou Arquitetura Orientada a Serviços. Este projeto de conclusão de curso tem como objetivo a construção de uma arquitetura utilizando este modelo para integrar aplicações resultantes de orientações de TCC e atividades realizadas no âmbito de laboratórios de pesquisa e desenvolvimento de \textit{software} na Universidade de Brasília. A compilação dos resultados desse trabalho dá origem à um sistema heterogêneo com características de uma plataforma web. Para a construção da plataforma baseada no modelo SOA, foi utilizado um barramento de serviços, no qual as aplicações e os serviços estão interligados. Este barramento é o ator responsável pelo roteamento, formatação e transformação de dados trocados via mensagens entre os componentes da arquitetura.

 \vspace{\onelineskip}
    
 \noindent
 \textbf{Palavras-chaves}: Arquitetura de Software. Arquitetura Orientada a Serviços. Ambiente Virtual Integrado. Sistemas heterogêneos. Barramento de Serviços.
\end{resumo}
